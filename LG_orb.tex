\documentclass[12pt]{scrartcl}
\usepackage{a4}
\usepackage{amsthm}
\usepackage{amsmath}
\usepackage{amssymb}
\usepackage{amsfonts}
\usepackage{mathrsfs}
\usepackage{dsfont}
\usepackage{latexsym}
\usepackage{color}
\usepackage{bbm,exscale}
\definecolor{Myblue}{rgb}{0,0,0.6}
\usepackage[a4paper,colorlinks,citecolor=Myblue,linkcolor=Myblue,urlcolor=Myblue,pdfpagemode=None]{hyperref}
\usepackage[square,numbers,sort&compress]{natbib}
\usepackage[all,cmtip]{xy}
\usepackage{tikz}

  \tolerance 1414
  \hbadness 1414
  \hfuzz 0.3pt
  \widowpenalty=10000
  \vfuzz \hfuzz
  \raggedbottom
  
\def\nicecolourscheme{\shadedraw[top color=blue!5, bottom color=blue!35, draw=blue!50!black, dashed]}
\def\boringcolourscheme{\draw[fill=blue!20, dashed]}

\newcommand{\D}{\text{d}}
\newcommand{\E}{\text{e}}
\newcommand{\I}{\text{i}}
\newcommand{\C}{\mathds{C}}
\newcommand{\M}{\mathds{M}}
\newcommand{\N}{\mathds{N}}
\newcommand{\Q}{\mathds{Q}}
\newcommand{\R}{\mathds{R}}
\newcommand{\Z}{\mathds{Z}}
\def\1{\ifmmode\mathrm{1\!l}\else\mbox{\(\mathrm{1\!l}\)}\fi}
\newcommand{\one}{\mathbbm{1}}
\newcommand{\be}{\begin{equation}}
\newcommand{\ee}{\end{equation}}
\newcommand{\bes}{\begin{equation*}}
\newcommand{\ees}{\end{equation*}}

\newcommand{\sVir}{\mathsf{sVir}}
\newcommand{\MF}{\operatorname{MF}_{\operatorname{bi}}}
\newcommand{\MFW}{\operatorname{MF}_{\operatorname{bi}}(W)}
\newcommand{\MFR}{\operatorname{MF}^\text{R}_{\operatorname{bi}}}
\newcommand{\DG}{\operatorname{DG}_{\operatorname{bi}}}
\newcommand{\DGW}{\operatorname{DG}_{\text{bi}}(W)}
\newcommand{\DGR}{\operatorname{DG}^\text{R}_{\text{bi}}}
\newcommand{\tc }{\otimes_\C}
\newcommand{\tr}{\otimes_R}
\newcommand{\id}{\text{id}}
\newcommand{\KMF}{K_{0}(\operatorname{MF}_{\text{bi}}}
\newcommand{\Ext}{\operatorname{Ext}}
\newcommand{\Hom}{\operatorname{Hom}}
\newcommand{\End}{\operatorname{End}}
\newcommand{\ev}{\operatorname{ev}}
\newcommand{\tev}{\widetilde{\operatorname{ev}}}
\newcommand{\coev}{\operatorname{coev}}
\newcommand{\tcoev}{\widetilde{\operatorname{coev}}}
\newcommand{\Ga}[1]{\Gamma_{\hspace{-2pt}#1}}

\newcommand\nxt{\noindent\raisebox{.08em}{\rule{.44em}{.44em}}\hspace{.4em}}
\newcommand\arxiv[2]      {\href{http://arXiv.org/abs/#1}{#2}}
\newcommand\doi[2]        {\href{http://dx.doi.org/#1}{#2}}
\newcommand\httpurl[2]    {\href{http://#1}{#2}}

\renewcommand{\labelenumi}{(\roman{enumi})}

\allowdisplaybreaks

\deffootnote[1em]{1em}{1em}
{\textsuperscript{\thefootnotemark}}

\theoremstyle{definition}
\newtheorem{definition}{Definition}
\newtheorem{proposition}[definition]{Proposition}
\newtheorem{theorem}[definition]{Theorem}
\newtheorem{lemma}[definition]{Lemma}
\newtheorem{remark}[definition]{Remark}
\newtheorem{remarks}[definition]{Remarks}
\newtheorem{conjecture}[definition]{Conjecture}

\numberwithin{equation}{section}
\numberwithin{definition}{section}
\numberwithin{figure}{section}


\newcommand\void[1]{}

\begin{document}

\title{
Orbifold completion of defect bicategories
\\[1em]
Generalised orbifolds and Frobenius algebras -- with an application to Landau-Ginzburg models
\\[1em]
Frobenius algebras and Landau-Ginzburg orbifolds
\\[1em]
Bicategory of Landau-Ginzburg orbifolds
\\[1em]
Orbifold completion of defect bicategories with an application to Landau-Ginzburg models
\\[1em]
Generalised orbifolds of Landau-Ginzburg models
\\[1em]
CARGO - canonical algebras for rigid generalised orbifolds
\\[1em]
etc.
}
\author{Nils Carqueville$^*$ \quad Ingo Runkel$^\dagger$
\\[0.5cm]
 \normalsize{\tt \href{mailto:nils.carqueville@physik.uni-muenchen.de}{nils.carqueville@physik.uni-muenchen.de}} \quad
  \normalsize{\tt \href{mailto:ingo.runkel@uni-hamburg.de}{ingo.runkel@uni-hamburg.de}}\\[0.1cm]
  {\normalsize\slshape $^*$Arnold Sommerfeld Center for Theoretical Physics, }\\[-0.1cm]
  {\normalsize\slshape LMU M\"unchen, Theresienstra\ss e~37, D-80333 M\"unchen}\\[-0.1cm]
  {\normalsize\slshape $^*$Excellence Cluster Universe, Boltzmannstra\ss e~2, D-85748 Garching}\\[0.1cm]
  {\normalsize\slshape $^\dagger$Department Mathematik, Universit\"{a}t Hamburg, }\\[-0.1cm]
  {\normalsize\slshape Bundesstra\ss e 55, D-20146 Hamburg}\\[-0.1cm]
}
\date{}
\maketitle

\vspace{-11.8cm}
\hfill {\scriptsize Hamburger Beitr\"age zur Mathematik XXX}

\vspace{-1.0cm}

\hfill {\scriptsize ZMP-HH/XXX}

\vspace{-1.0cm}

\hfill {\scriptsize LMU-ASC XXX}


\vspace{12cm}

\begin{abstract}
category ... category ... category ...
\end{abstract}

\newpage

\tableofcontents


\section{Introduction and summary}\label{introduction}

\begin{verbatim}
   * ...
   * inflating new domains into worldsheets 
   * sketch of TFT_orb from TFT
   * summary of results
   * ...
\end{verbatim}

\section{Two-dimensional topological field theory with defects}

\subsection{TFT with defects as symmetric monoidal functors}

\subsection{Orbifold TFTs}

\subsection{Bicategory of world sheet phases}


\section{Algebraic Background}

\begin{verbatim}
  * motivational paragraph/remark: it's all good, don't be scared of these diagrams; 
  they're easy, intuitional and efficient; intepretation in terms of defects, boundary 
  conditions etc.; read FRS
\end{verbatim}


\subsection{Bicategories with adjoints}

\begin{verbatim}
   * bicategories with adjoints (mostly use string diagrams after initial definition of Zorro)
   * left & right traces, quantum dimensions etc. 
\end{verbatim}

\subsection{Algebras and bimodules}

algebras

Frobenius algebras

Frobenius algebras with trace pairing

Prop.

bimodules, bimodule maps

\subsection{Tensor products}

\begin{verbatim}
   * tensor product over algebras via coequaliser property; exists if idempotents split
   * projector from all 2-morphisms to bimodule maps
\end{verbatim}

\section{Orbifold bicategory}

\subsection{Definition of $\mathcal{D}_\mathrm{orb}$}

\begin{verbatim}
   * Let B be bicategory with adjoints; pivotality for End(W); Hom(V,W) idempotent complete. 
   * Definition of B_orb. 
   * Remark: TFT_orb also describes TFT with defects (not known what B, B_orb need to satisfy)
\end{verbatim}

$\mathcal{D}$ is fully embedded in $D_\mathrm{orb}$ via U to $(U,1)$

Maybe comment on Calin's suggestions to generalise $D_\mathrm{orb}$, possible relation to Pronk's Etendues and stacks as bicategories of fractions and later work

\subsection{$(\mathcal{D}_\mathrm{orb})_\mathrm{orb}$ is equivalent to $\mathcal{D}_\mathrm{orb}$}

\begin{verbatim}
   * TODO: B_orb versus (B_orb)_orb
\end{verbatim}

\subsection{Equivalences in $\mathcal{D}_\mathrm{orb}$ and invertible quantum dimensions}

\begin{verbatim}
   * Theorem: A = X^ x X is ssFA if X has invertible qdims (otherwise only symmetric FA)
   * Theorem: X x_A X^ = Id
   * Theorem: X and X^ are mutually inverse 1-morphisms in B_orb
\end{verbatim}

\subsection{Non-degenerate pairings}

\subsection{Comments on examples?}

LG models, see next two sections

maybe include comments about A and B models as pivotal bicategories? (Caldararu-Willerton on B-model, Wehrheim et al.~on A-model; look e.\,g.~at \begin{verbatim}
http://math.mit.edu/~katrin/papers/quiltfloer.pdf
\end{verbatim})



\section{Bicategory of Landau-Ginzburg models}

\subsection{Definition of $\mathcal{LG}_k$ and adjoints}

\begin{verbatim}
       # definition of LG_k, evals and coevals
       # Theorem [PV]: sphere correltator = disk correlator with Delta_W as boundary condition 
       # formulas for defect action etc. 
       # either enhance discussion by signs and shifts OR restrict to same parity of variables
\end{verbatim}

\subsection{Landau-Ginzburg orbifolds}

Show that $LG_k$ has required properties

idempotents split

(reduced?) pivotal

nondegenerate : need to pass to sub-bicategory of even or odd number of variables
       
\subsection{Graded matrix factorisations and central charge}

briefly explain matrix factorisations with $U(1)$-charge (just in matrix notation, not the equivariant category setup from our first paper). Define central charge in terms of charges of potential. 

Prop:
for matrix factorisations with $U(1)$-charge can have defects of invertible quantum dimension only if c1=c2.

\subsection{Open-closed TFTs from orbifolds}

\begin{verbatim}
   * Theorem: bulk and boundary metric in B_orb are nondeg. if B has nondeg. <->: End(I_W)-->C. 
   * Theorem: if End_B(W) describes o/cTFT and lasso lemma holds, then End_{B_orb}((W,A)) is o/cTFT.
\end{verbatim}


\section{Examples of Landau-Ginzburg orbifolds}

\subsection{Equivariant matrix factorisations}

\begin{verbatim}
       # A = Delta! 
       # A = sum_g Delta_G is ssFA; hmf(W)^G = mod(A)
\end{verbatim}

Thus proved Cardy condition for G-equivariant matrix factorisation as special case of Cardy for $LG_{orb}$

\subsection{Kn\"orrer periodicity}

\begin{verbatim}
       # X = Delta x (0 v // u 0) => Knörrer periodicity
\end{verbatim}

\subsection{Potentials $x^{2d}+y^2$ and $x^{d} + x y^2$}

\begin{verbatim}
       # X = bla => hmf(W_D) = hmf(W_A)^Z2
           ** TODO: X^ x X = Delta + P_d/2[1]
           ** Point out that that we don't have to check End_AA(A) = Jac etc.
       # comments on how to construct X with qdim != 0? E-type example more systematically
\end{verbatim}


\begin{thebibliography}{BHLS}

\bibitem[CR]{cr0909.4381}
N.~Carqueville and I.~Runkel,
{\it On the monoidal structure of matrix bi-factorisations}, \doi{10.1088/1751-8113/43/27/275401}{J. Phys. A: Math. Theor. \textbf{43} (2010), 275401},
\arxiv{0909.4381}{[0909.4381 [math-ph]]}.


\end{thebibliography}


\end{document}

